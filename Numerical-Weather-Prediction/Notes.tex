%%%%%%%%%%%%%%%%%%%%%%%%%%%%%%%%%%%%%%%%%%%%%%%%%%%%%%%%%%%%%%%%%%%%%%%%%%%%%%%%%%%%%
%                        Author: Harshit Prashant Dhanwalkar                        %
%%%%%%%%%%%%%%%%%%%%%%%%%%%%%%%%%%%%%%%%%%%%%%%%%%%%%%%%%%%%%%%%%%%%%%%%%%%%%%%%%%%%%

%-------------------------------------------------------------------------------------
%                    PACKAGES AND OTHER DOCUMENT CONFIGURATIONS                      %
%-------------------------------------------------------------------------------------

\documentclass[fleqn,10pt]{SelfArx} % Document font size and equations flushed left
\usepackage[english]{babel} 
\usepackage{enumitem}

\usepackage{tikz}
\usetikzlibrary{trees, positioning, shapes}
\usepackage{tikz-3dplot}
\tikzstyle{vertex}=[draw,fill=black!15,circle,minimum size=20pt,inner sep=0pt]
\tikzstyle{selected edge} = [draw,line width=5pt,-,red!50]

\usepackage{amsmath}
\usepackage{array} % for tables

\usepackage{pgfplots}
\pgfplotsset{compat=1.18}
\tdplotsetmaincoords{60}{115}
\pgfplotsset{compat=newest}

%-------------------------------------------------------------------------------------
%                                       COLUMNS                                      %
%-------------------------------------------------------------------------------------
\setlength{\columnsep}{0.55cm} % Distance between the two columns of text
\setlength{\fboxrule}{0.75pt} % Width of the border around the abstract

%-------------------------------------------------------------------------------------
%                                        COLORS                                      %
%-------------------------------------------------------------------------------------
\definecolor{color1}{RGB}{0,0,90} % Color of the article title and sections
\definecolor{color2}{RGB}{0,20,20} % Color of the boxes behind the abstract and headings

%-------------------------------------------------------------------------------------
%                                       EQUATIONS                                    %
%-------------------------------------------------------------------------------------
\usepackage{cancel} % for crossing the word showing it is cancelled or is zero

%-------------------------------------------------------------------------------------
%                                     EQUATIONLINKS                                  %
%-------------------------------------------------------------------------------------
\newcommand{\myeqref}[1]{\textcolor{blue}{\textup{(\getrefnumber{#1})}}}

%-------------------------------------------------------------------------------------
%                                       HYPERLINKS                                   %
%-------------------------------------------------------------------------------------

\usepackage{xcolor}
\usepackage{hyperref}
\usepackage{footnote}

%\newcommand{\myhref}[2]{\href{#1}{\textcolor{blue}{#2}}}
\newcommand{\myhref}[2]{%
  \href{#1}{\textcolor{blue}{#2}}%
  \footnote{\url{#1}}%
}

\usepackage{cleveref}
% Customize cleveref to use "Eq." for equations
\crefname{equation}{Eq.}{Eq.}
\Crefname{equation}{Eq.}{Eq.}

\hypersetup{
	hidelinks,
	colorlinks,
	breaklinks=true,
	urlcolor=color2,
	citecolor=color1,
	linkcolor=color1,
	bookmarksopen=false,
	pdftitle={Title},
	pdfauthor={Author},
}

% ------------------------------------------------------------------------------------
%                                       CUSTOM  SYMBOLS                              %
%-------------------------------------------------------------------------------------
\newcommand{\zbar}{\raisebox{0.2ex}{--}\kern-0.6em Z}

%-------------------------------------------------------------------------------------
%                                       ARTICLE INFORMATION                          %
%-------------------------------------------------------------------------------------
\JournalInfo{Dual Degree Engineering Physics, 8$^{th}$ Semester, 2024} % Journal information
\Archive{Mtech, Earth System Sciences (ESS), 1$^{st}$ year} % Additional notes (e.g. copyright, DOI, review/research article)

\PaperTitle{Lecture Notes on Numerical Weather Prediction} % Article title

\Authors{Harshit Prashant Dhanwalkar (SC21B164)\textsuperscript{1}*} % Authors
\affiliation{\textsuperscript{1}\textit{MTech, Earth System Sciences (ESS), 1$^{st}$ year, Department of Physics, Indian Institute Of Spacescience and Technology (IIST)}} % Author affiliation
\affiliation{*\textbf{email}: harshitpd1729@gamil.com} % Corresponding author

\Keywords{} % Keywords - if you don't want any simply remove all the text between the curly brackets
\newcommand{\keywordname}{Keywords} % Defines the keywords heading name

%-------------------------------------------------------------------------------------
%                                           ABSTRACT                                 %
%-------------------------------------------------------------------------------------
\Abstract{Notes of Lectures and addional information from books.}

%-------------------------------------------------------------------------------------
%                                            DOCUMENT                                %
%-------------------------------------------------------------------------------------
\begin{document}
\maketitle % Output the title and abstract box
\clearpage
\tableofcontents % Output the contents section
\clearpage
\thispagestyle{empty} % Removes page numbering from the first page
\clearpage

%-------------------------------------------------------------------------------------
%                                       DOCUMENT CONTENTS                           %
%-------------------------------------------------------------------------------------
%\addcontentsline{toc}{section}{Introduction} % Adds this section to the table of contents
%-------------------------------------------------------------------------------------
\section{Lecture 1 06/01/2025}
Numerical Weathering Problem (NWP) was first proposed by Bjerkives around 1900. It is mathematical initial value problem (IVP).

Initial value Problem (IVP) $\rightarrow$ simple pendulum.
\begin{align}
	\ddot{\theta} + \omega^2 \theta          & = 0 \label{eq:simplependulum1}                                          \\
	\frac{d^2\theta}{dt^2} + \omega^2 \theta & = 0 \label{eq:simplependulum2}                                          \\
	\theta(t)                                & = A \cos(\omega t) + B \sin(\omega t) \label{eq:simplependulumsolution}
\end{align}

Eq.\eqref{eq:simplependulum1} and \eqref{eq:simplependulum2} are second order linear ordinary differential equation, whose solution Eq.\eqref{eq:simplependulumsolution} has 2 constants of integration $A$ and $B$. Here $\theta$ and $t$ are the dependent and independent variable since Eq.\eqref{eq:simplependulum1} and \eqref{eq:simplependulum2} have only one independent variable.

Values of $A$ and $B$ will depend on initial condition.

Since ODE is second order, 2 initial condition are needed at initial time, say $t=0$. Which are:
\begin{align}
	\left.
	\begin{aligned}
		\theta(t=0) = 1 \\
		\frac{\theta(t=0)}{dt} = 0
	\end{aligned}
	\right\} \quad \quad & \label{eq:att=0}
\end{align}

Eq.\eqref{eq:simplependulum2} and initial conditions Eq.\eqref{eq:att=0} are together called \textbf{Mathematical IVP}. For any physical system the following two requirements are needed:
\begin{enumerate}[noitemsep]
	\item The equation (ODE or PDE) that governs the evolution of the above system.
	\item The initial state of the system.
\end{enumerate}

7 independent variables \textbf{(u,v,w,T,$\boldsymbol{\rho}$,p,q)}.

Surface area of Earth = $4\pi R^2$ = $4\pi (6.37 \times 10^{12})$ $\approx 5.1 \times 10^{14}$ m$^2$
%-------------------------------------------------------------------------------------
\clearpage
%-------------------------------------------------------------------------------------
\section{Lecture 2 07/01/2025}
7 independent variables \textbf{(u,v,w,T,$\boldsymbol{\rho}$,p,q)} therefore we need 7 Governing equations (system of 7 coupled non-linear partial differential equations):
\begin{enumerate}[noitemsep]
	\item Conservation of masss (continuity equation).
	\item Conservation of momentum in rotating frame of refrence (3 scalar equations, one each corresponding to scalar component of velocity).
	\item Conservation of energy (Thermodynamic energy equation).
	\item Conservation of moisture (moisture continuity equation).
	\item Equation of state (Ideal gas equation).
\end{enumerate}

Euler discription of fluid motion is more convinent becasue of dependance on time and above 7 equations.

Total advective and convective time of lagrangian is given by:
\begin{align*}
	\underbrace{\frac{DT}{Dt}}_\text{Lagrangian Derivative} & = \underbrace{\underbrace{\frac{\partial T}{\partial t}}_\text{Local derivative} + \underbrace{\vec{V}\cdot \nabla T}_\text{Advective Term}}_\text{Euler Derivative} \\
	\frac{DT}{Dt}                                           & = \frac{\partial T}{\partial t} + u\frac{\partial T}{\partial x} + v\frac{\partial T}{\partial y} + w\frac{\partial T}{\partial z} \tag{5} \label{eq:lageuleq}
\end{align*}

Using first law of Thermodynamics, Rate of heat is given by:
\begin{align*}
	d\dot{q}         & = d\dot{u} + d\dot{w}                 \\
	\frac{DU}{Dt}    & = \frac{Dq}{Dt} - \frac{Dw}{Dt}       \\
	C_v\frac{DT}{Dt} & = \frac{Dq}{Dt} - p\frac{D\alpha}{Dt}
\end{align*}

where $\frac{Dq}{Dt}$ is rate at which heating of air parcel due to non-adiabatic process, this change can happen via radiation, convection, conduction, latent heat while phase change.

\begin{align*}
	\frac{DU}{Dt} & = \vec{F}_\text{net} + \vec{F}_\text{coriolis} \tag{6} \label{eq:convective_derivative}
\end{align*}

This above Eq.\eqref{eq:convective_derivative} is convective derivative equation involving non-linear terms (i.e. $u\frac{\partial T}{\partial x}$, $v\frac{\partial T}{\partial y}$, $w\frac{\partial T}{\partial z}$).

Continuity equation:
\begin{align*}
	\frac{1}{\rho} \frac{D\rho}{Dt} + \nabla\cdot\vec{V} = 0  \tag{7} \label{eq:continuity_eq}
\end{align*}

Let grid of following resolutions:
\begin{itemize}[noitemsep]
	\item $1^\circ\times 1^\circ$ $\rightarrow$ $3\times 10^6$ grid cells $\therefore$ no. of variables $\rightarrow$ $7 \times 3\times 10^6$.
	\item $5^\circ\times 5^\circ$ $\rightarrow$ $1.3\times 10^5$ grid cells $\therefore$ no. of variables $\rightarrow$ $7 \times 1.3\times 10^5$.
	\item $20^\circ\times 20^\circ$ $\rightarrow$ $9\times 10^3$ grid cells $\therefore$ no. of variables $\rightarrow$ $7 \times 9\times 10^3$.
	\item $25^\circ\times 25^\circ$ $\rightarrow$ $6\times 10^3$ grid cells $\therefore$ no. of variables $\rightarrow$ $7 \times 6\times 10^3$.
\end{itemize}
These are even larger than entire country, which means that we can't above to find the change of varibles with these grids. This we don't have a way to determine initial condition, if we try to use interpolation, it will cause errors which will grow with time since atmosphere is chaotic and dynamic system.

% \begin{tikzpicture}[tdplot_main_coords, scale = 2]
% 	% Create a point (P)
% 	% \coordinate (P) at ({1/sqrt(3)},{1/sqrt(3)},{1/sqrt(3)});
% 	% Draw shaded circle
% 	\shade[ball color = lightgray,
% 		opacity = 0.5
% 	] (0,0,0) circle (1cm);
% 	% draw arcs 
% 	\tdplotsetrotatedcoords{0}{0}{0};
% 	\draw[dashed,
% 		tdplot_rotated_coords,
% 		gray
% 	] (0,0,0) circle (1);
% 	\tdplotsetrotatedcoords{90}{90}{90};
% 	\draw[dashed,
% 		tdplot_rotated_coords,
% 		gray
% 		%] (1,0,0) arc (0:180:1);
% 	] (0,0,0) circle (1);
% 	\tdplotsetrotatedcoords{45}{45}{45};
% 	\draw[dashed,
% 		tdplot_rotated_coords,
% 		gray
% 		%] (1,0,0) arc (0:180:1);
% 	] (0,0,0) circle (1);
% 	% Projection of the point on X and y axes
% 	%\draw[thin, dashed] (P) --++ (0,0,{-1/sqrt(3)});
% 	% \draw[thin, dashed] ({1/sqrt(3)},{1/sqrt(3)},0) --++
% 	% (0,{-1/sqrt(3)},0);
% 	% \draw[thin, dashed] ({1/sqrt(3)},{1/sqrt(3)},0) --++
% 	% ({-1/sqrt(3)},0,0);
% 	% Axes in 3 d coordinate system
% 	\draw[-stealth] (0,0,0) -- (1.80,0,0)
% 	node[below left] {$x$};
% 	\draw[-stealth] (0,0,0) -- (0,1.30,0)
% 	node[below right] {$y$};
% 	\draw[-stealth] (0,0,0) -- (0,0,1.30)
% 	node[above] {$z$};
% 	\draw[dashed, gray] (0,0,0) -- (-1,0,0);
% 	\draw[dashed, gray] (0,0,0) -- (0,-1,0);
% 	% Line from the origin to (P)
% 	% \draw[thick, -stealth] (0,0,0) -- (P) node[right] {$P$};
% 	% % Add small circle at (P)
% 	% \draw[fill = lightgray!50] (P) circle (0.5pt);
% \end{tikzpicture}

%-------------------------------------------------------------------------------------
\newcommand\pgfmathsinandcos[3]{%
	\pgfmathsetmacro#1{sin(#3)}%
	\pgfmathsetmacro#2{cos(#3)}%
}
\newcommand\LongitudePlane[3][current plane]{%
	\pgfmathsinandcos\sinEl\cosEl{#2} % elevation
	\pgfmathsinandcos\sint\cost{#3} % azimuth
	\tikzset{#1/.style={cm={\cost,\sint*\sinEl,0,\cosEl,(0,0)}}}
}
\newcommand\LatitudePlane[3][current plane]{%
	\pgfmathsinandcos\sinEl\cosEl{#2} % elevation
	\pgfmathsinandcos\sint\cost{#3} % latitude
	\pgfmathsetmacro\yshift{\cosEl*\sint}
	\tikzset{#1/.style={cm={\cost,0,0,\cost*\sinEl,(0,\yshift)}}} %
}
\newcommand\DrawLongitudeCircle[2][1]{
	\LongitudePlane{\angEl}{#2}
	\tikzset{current plane/.prefix style={scale=#1}}
	% angle of "visibility"
	\pgfmathsetmacro\angVis{atan(sin(#2)*cos(\angEl)/sin(\angEl))} %
	\draw[current plane,thin,black] (\angVis:1) arc (\angVis:\angVis+180:1);
	\draw[current plane,thin,dashed] (\angVis-180:1) arc (\angVis-180:\angVis:1);
}%this is fake: for drawing the grid
\newcommand\DrawLongitudeCirclered[2][1]{
	\LongitudePlane{\angEl}{#2}
	\tikzset{current plane/.prefix style={scale=#1}}
	% angle of "visibility"
	\pgfmathsetmacro\angVis{atan(sin(#2)*cos(\angEl)/sin(\angEl))} %
	\draw[current plane,red,thick] (150:1) arc (150:180:1);
	%\draw[current plane,dashed] (-50:1) arc (-50:-35:1);
}%for drawing the grid
\newcommand\DLongredd[2][1]{
	\LongitudePlane{\angEl}{#2}
	\tikzset{current plane/.prefix style={scale=#1}}
	% angle of "visibility"
	\pgfmathsetmacro\angVis{atan(sin(#2)*cos(\angEl)/sin(\angEl))} %
	\draw[current plane,black,dashed, ultra thick] (150:1) arc (150:180:1);
}
\newcommand\DLatred[2][1]{
	\LatitudePlane{\angEl}{#2}
	\tikzset{current plane/.prefix style={scale=#1}}
	\pgfmathsetmacro\sinVis{sin(#2)/cos(#2)*sin(\angEl)/cos(\angEl)}
	% angle of "visibility"
	\pgfmathsetmacro\angVis{asin(min(1,max(\sinVis,-1)))}
	\draw[current plane,dashed,black,ultra thick] (-50:1) arc (-50:-35:1);
}
\newcommand\fillred[2][1]{
	\LongitudePlane{\angEl}{#2}
	\tikzset{current plane/.prefix style={scale=#1}}
	% angle of "visibility"
	\pgfmathsetmacro\angVis{atan(sin(#2)*cos(\angEl)/sin(\angEl))} %
	\draw[current plane,red,thin] (\angVis:1) arc (\angVis:\angVis+180:1);
}
\newcommand\DrawLatitudeCircle[2][1]{
	\LatitudePlane{\angEl}{#2}
	\tikzset{current plane/.prefix style={scale=#1}}
	\pgfmathsetmacro\sinVis{sin(#2)/cos(#2)*sin(\angEl)/cos(\angEl)}
	% angle of "visibility"
	\pgfmathsetmacro\angVis{asin(min(1,max(\sinVis,-1)))}
	\draw[current plane,thin,black] (\angVis:1) arc (\angVis:-\angVis-180:1);
	\draw[current plane,thin,dashed] (180-\angVis:1) arc (180-\angVis:\angVis:1);
}%Defining functions to draw limited latitude circles (for the red mesh)
\newcommand\DrawLatitudeCirclered[2][1]{
	\LatitudePlane{\angEl}{#2}
	\tikzset{current plane/.prefix style={scale=#1}}
	\pgfmathsetmacro\sinVis{sin(#2)/cos(#2)*sin(\angEl)/cos(\angEl)}
	% angle of "visibility"
	\pgfmathsetmacro\angVis{asin(min(1,max(\sinVis,-1)))}
	%\draw[current plane,red,thick] (-\angVis-50:1) arc (-\angVis-50:-\angVis-20:1);
	\draw[current plane,red,thick] (-50:1) arc (-50:-35:1);
}

\tikzset{%
	>=latex,
	inner sep=0pt,%
	outer sep=2pt,%
	mark coordinate/.style={inner sep=0pt,outer sep=0pt,minimum size=3pt,
			fill=black,circle}%
}
\pagestyle{empty}

\begin{figure}[ht!]
	\centering
	\begin{tikzpicture}[scale=0.5,every node/.style={minimum size=1cm}]
		%% some definitions
		\def\R{4} % sphere radius
		\def\angEl{25} % elevation angle
		\def\angAz{-100} % azimuth angle
		\def\angPhiOne{-50} % longitude of point P
		\def\angPhiTwo{-35} % longitude of point Q
		\def\angBeta{30} % latitude of point P and Q

		%% working planes
		\pgfmathsetmacro\H{\R*cos(\angEl)} % distance to north pole
		\LongitudePlane[xzplane]{\angEl}{\angAz}
		\LongitudePlane[pzplane]{\angEl}{\angPhiOne}
		\LongitudePlane[qzplane]{\angEl}{\angPhiTwo}
		\LatitudePlane[equator]{\angEl}{0}
		\fill[ball color=white!10] (0,0) circle (\R); % 3D lighting effect
		\coordinate (O) at (0,0);
		\coordinate[mark coordinate] (N) at (0,\H);
		\coordinate[mark coordinate] (S) at (0,-\H);
		\path[xzplane] (\R,0) coordinate (XE);

		%defining points outsided the area bounded by the sphere
		\path[pzplane] (\R,0) coordinate (PE);
		\path[qzplane] (\angBeta:\R) coordinate (Q);
		\path[qzplane] (\angBeta:\R) coordinate (Qd);%sfora di una quantità pari a 10 dopo la sfera
		\path[qzplane] (\R,0) coordinate (QE);

		\DrawLongitudeCircle[\R]{\angPhiOne} % pzplane
		\DrawLongitudeCircle[\R]{\angPhiTwo} % qzplane
		\DrawLatitudeCircle[\R]{\angBeta}
		\DrawLatitudeCircle[\R]{0} % equator
		%labelling north and south
		\node[above=1pt] at (N) {$\mathbf{N}$};
		\node[below=1pt] at (S) {$\mathbf{S}$};

		\draw[-,dashed, thick] (N) -- (S);

		\path[xzplane] (0:\R) node[below] {$$};
		\path[xzplane] (\angBeta:\R) node[below left] {$$};
		\foreach \t in {0,2,...,30} { \DrawLatitudeCirclered[\R]{\t} }
		\foreach \t in {130,133,...,145} { \DrawLongitudeCirclered[\R]{\t} }

		%drawing grids on the spere invoking DLongredd and DrawLongitudeCirclered
		\foreach \t in {130,145,...,145} { \DLongredd[\R+3]{\t} }
		\foreach \t in {130,133,...,145} { \DrawLongitudeCirclered[\R+3]{\t} }
		\foreach \t in {0,30,...,30} { \DLatred[\R+3]{\t} }
		\foreach \t in {0,2,...,30} { \DrawLatitudeCirclered[\R+3]{\t} }
	\end{tikzpicture}
	\caption{Figure showing the grid}
\end{figure}
%-------------------------------------------------------------------------------------
\clearpage
%-------------------------------------------------------------------------------------
\section{Lecture 3 08/01/2025}
\begin{align*}
	u & = \bar{u} + u' \tag{8} \label{eq:u}
\end{align*}

Here, $u$ is the velocity field, which is decomposed into a mean component $\bar{u}$ and a fluctuating component $u'$.

Navier-Stokes Equation
The general Navier-Stokes equation is given by:
\begin{align*}
	\frac{\partial u}{\partial t} + u \frac{\partial u}{\partial x} + v \frac{\partial u}{\partial y} & + w \frac{\partial u}{\partial z} - f v = \frac{1}{\rho} \frac{\partial \bar{P}}{\partial x} \\ & + \gamma \left(\frac{\partial^2 u}{\partial x^2} + \frac{\partial^2 u}{\partial y^2}+ \frac{\partial^2 u}{\partial z^2}\right) \tag{9} \label{eq:Navier-stokes}
\end{align*}

Reynolds-Averaged Navier-Stokes (RANS) Equation
Applying Reynolds decomposition ($u = \bar{u} + u'$) and averaging leads to the RANS equation:
\begin{align*}
	\frac{\partial \bar{u}}{\partial t} & + \bar{u} \frac{\partial \bar{u}}{\partial x} + \bar{v} \frac{\partial \bar{u}}{\partial y} + \bar{w} \frac{\partial \bar{u}}{\partial z} - f\bar{v} = \frac{1}{\rho} \frac{\partial \bar{P}}{\partial x}                                                                                               \\
	                                    & + \gamma \left(\frac{\partial^2 \bar{u}}{\partial x^2}+ \frac{\partial^2 \bar{u}}{\partial y^2}+ \frac{\partial^2 \bar{u}}{\partial z^2}\right)                                                                                                                                                         \\
	                                    & + \underbrace{\frac{1}{\rho} \left(\frac{\partial \left(-\rho \overline{u'u'} \right)}{\partial x} + \frac{\partial \left(-\rho \overline{u'v'} \right)}{\partial y} + \frac{\partial \left(-\rho \overline{u'w'} \right)}{\partial z}\right)}_{\text{Reynolds stress tensor}} \tag{10} \label{eq:rans}
\end{align*}

The Reynolds stress tensor represents the transport of momentum due to turbulent fluctuations.

Nonlinear Term Expansion
Expanding the nonlinear term $u \frac{\partial u}{\partial x}$ using Reynolds decomposition:
\begin{align*}
	u \frac{\partial u}{\partial x} & = (\bar{u}+u') \frac{\partial(\bar{u}+u')}{\partial x}                                                                                                              \\
	                                & = \bar{u} \frac{\partial \bar{u}}{\partial x}+ u' \frac{\partial \bar{u}}{\partial x} + \bar{u} \frac{\partial u'}{\partial x} + u' \frac{\partial u'}{\partial x}.
\end{align*}

Appliing Reynolds averaging rules:
\begin{align*}
	\overline{u} & = \overline{\bar{u} + u'}                                            \\
	\overline{u} & = \overline{\bar{u}} + \overline{u'}                                 \\
	\overline{u} & = \bar{u} + \overline{u'} \quad \Rightarrow \quad \overline{u'} = 0.
\end{align*}

Thus, the fluctuating component $u'$ averages out to zero over time, leaving only the mean component $\bar{u}$ in the averaged equations.

We have,
\begin{align*}
	\frac{\partial u}{\partial x} + \frac{\partial v}{\partial y} + \frac{\partial w}{\partial z} & =0 \tag{11} \label{eq:velocity}
\end{align*}
Substituting $u$, $v$ and $w$ in above Eq.\eqref{eq:velocity}, we get:

\begin{align*}
	\overline{\frac{\partial (\bar{u}+u')}{\partial x} + \frac{\partial (\bar{v}+v')}{\partial y} + \frac{\partial (\bar{w}+w')}{\partial z}} & =0 \\
	\frac{\partial \bar{u}}{\partial x} + \frac{\partial \bar{v}}{\partial y} + \frac{\partial \bar{w}}{\partial z}                           & =0
\end{align*}

The term $\frac{\partial(\overline{u'u'})}{\partial t}$ represents the rate of change of kinetic energy per unit mass due to turbulent fluctuations. It can be expressed as:
\begin{align*}
	\frac{\partial (\overline{u'u'})}{\partial t}
	 & = \frac{\partial \left( \rho \overline{u'v'} \right)}{\partial y} + \ldots
\end{align*}

This term involves higher-order correlations between velocity fluctuations, which complicates the equation system.

The closure problem arises in the Reynolds-Averaged Navier-Stokes (RANS) equations because the number of dependent variables (unknowns) exceeds the number of equations available. For instance:
- $\overline{u}$ is an unknown.
- $\overline{u'v'}$ (a Reynolds stress term) introduces additional unknowns.

To resolve this, closure modelsare used, which provide approximations for higher-order terms based on known variables.

For example, consider the term $\overline{u'w'}$. Using a simple closure model:
\begin{align*}
	\overline{u'w'} & = -k \frac{\partial \bar{u}}{\partial x},
\end{align*}
where $k$ is a proportionality constant (often related to eddy viscosity). Here, $\overline{u}$ is already an unknown, so no additional variables are introduced, avoiding further complexity.

This is an example of first-order closurewhere higher-order terms are approximated using first-order variables.

Types of Closure Models
\begin{enumerate}
	\item \textbf{First-Order Closure}:

	      - Simplifies higher-order terms using known variables and gradients (e.g., eddy viscosity models).

	      - Example: $\overline{u'w'} = -k \frac{\partial \bar{u}}{\partial x}$.

	      - Advantage: Computationally efficient but may lack accuracy in complex flows.
	\item \textbf{One-Point Closure}:

	      - Approximates turbulence at a single point using local flow properties.

	      - Example: Mixing length models, where turbulent viscosity is proportional to local shear.
	\item \textbf{Second-Order Closure}:

	      - Directly models second-order correlations like $\overline{u'u'}$ and $\overline{u'v'}$ by solving additional transport equations.

	      - Provides higher accuracy but increases computational cost.

	      - Example: Reynolds stress models (RSM), where additional equations are solved for Reynolds stresses.
\end{enumerate}
\end{document}
