%%%%%%%%%%%%%%%%%%%%%%%%%%%%%%%%%%%%%%%%%%%%%%%%%%%%%%%%%%%%%%%%%%%%%%%%%%%%%%%%%%%%%
%                        Author: Harshit Prashant Dhanwalkar                        %
%%%%%%%%%%%%%%%%%%%%%%%%%%%%%%%%%%%%%%%%%%%%%%%%%%%%%%%%%%%%%%%%%%%%%%%%%%%%%%%%%%%%%

%-------------------------------------------------------------------------------------
%                    PACKAGES AND OTHER DOCUMENT CONFIGURATIONS                      %
%-------------------------------------------------------------------------------------

\documentclass[fleqn,10pt]{SelfArx} % Document font size and equations flushed left
\usepackage[english]{babel} 
\usepackage{enumitem}

\usepackage{tikz}
\usetikzlibrary{trees, positioning, shapes}
\usepackage{tikz-3dplot}

\usepackage{amsmath}
\usepackage{array} % for tables

\usepackage{pgfplots}
\pgfplotsset{compat=1.18}

%-------------------------------------------------------------------------------------
%                                       COLUMNS                                      %
%-------------------------------------------------------------------------------------
\setlength{\columnsep}{0.55cm} % Distance between the two columns of text
\setlength{\fboxrule}{0.75pt} % Width of the border around the abstract

%-------------------------------------------------------------------------------------
%                                        COLORS                                      %
%-------------------------------------------------------------------------------------
\definecolor{color1}{RGB}{0,0,90} % Color of the article title and sections
\definecolor{color2}{RGB}{0,20,20} % Color of the boxes behind the abstract and headings

%-------------------------------------------------------------------------------------
%                                       EQUATIONS                                    %
%-------------------------------------------------------------------------------------
\usepackage{cancel} % for crossing the word showing it is cancelled or is zero

%-------------------------------------------------------------------------------------
%                                     EQUATIONLINKS                                  %
%-------------------------------------------------------------------------------------
\newcommand{\myeqref}[1]{\textcolor{blue}{\textup{(\getrefnumber{#1})}}}

%-------------------------------------------------------------------------------------
%                                       HYPERLINKS                                   %
%-------------------------------------------------------------------------------------

\usepackage{xcolor}
\usepackage{hyperref}
\usepackage{footnote}

%\newcommand{\myhref}[2]{\href{#1}{\textcolor{blue}{#2}}}
\newcommand{\myhref}[2]{%
  \href{#1}{\textcolor{blue}{#2}}%
  \footnote{\url{#1}}%
}

\usepackage{cleveref}
% Customize cleveref to use "Eq." for equations
\crefname{equation}{Eq.}{Eq.}
\Crefname{equation}{Eq.}{Eq.}

\hypersetup{
	hidelinks,
	colorlinks,
	breaklinks=true,
	urlcolor=color2,
	citecolor=color1,
	linkcolor=color1,
	bookmarksopen=false,
	pdftitle={Title},
	pdfauthor={Author},
}

% ------------------------------------------------------------------------------------
%                                       CUSTOM  SYMBOLS                              %
%-------------------------------------------------------------------------------------
\newcommand{\zbar}{\raisebox{0.2ex}{--}\kern-0.6em Z}

%-------------------------------------------------------------------------------------
%                                       ARTICLE INFORMATION                          %
%-------------------------------------------------------------------------------------
\JournalInfo{Dual Degree Engineering Physics, 8$^{th}$ Semester, 2024} % Journal information
\Archive{Mtech, Earth System Sciences (ESS), 1$^{st}$ year} % Additional notes (e.g. copyright, DOI, review/research article)

\PaperTitle{Lecture Notes on Numerical Weather Prediction} % Article title

\Authors{Harshit Prashant Dhanwalkar (SC21B164)\textsuperscript{1}*} % Authors
\affiliation{\textsuperscript{1}\textit{MTech, Earth System Sciences (ESS), 1$^{st}$ year, Department of Physics, Indian Institute Of Spacescience and Technology (IIST)}} % Author affiliation
\affiliation{*\textbf{email}: harshitpd1729@gamil.com} % Corresponding author

\Keywords{} % Keywords - if you don't want any simply remove all the text between the curly brackets
\newcommand{\keywordname}{Keywords} % Defines the keywords heading name

%-------------------------------------------------------------------------------------
%                                           ABSTRACT                                 %
%-------------------------------------------------------------------------------------
\Abstract{Notes of Lectures and addional information from books.}

%-------------------------------------------------------------------------------------
%                                            DOCUMENT                                %
%-------------------------------------------------------------------------------------
\begin{document}
\maketitle % Output the title and abstract box
\clearpage
\tableofcontents % Output the contents section
\clearpage
\thispagestyle{empty} % Removes page numbering from the first page
\clearpage

%-------------------------------------------------------------------------------------
%                                       DOCUMENT CONTENTS                           %
%-------------------------------------------------------------------------------------
%\addcontentsline{toc}{section}{Introduction} % Adds this section to the table of contents
%-------------------------------------------------------------------------------------
\section{Lecture 1 06/01/2025}
Numerical Weathering Problem (NWP) was first proposed by Bjerkives around 1900. It is mathematical initial value problem (IVP).

Initial value Problem (IVP) $\rightarrow$ simple pendulum.
\begin{align}
  \ddot{\theta} + \omega^2 \theta &= 0 \label{eq:simplependulum1} \\
  \frac{d^2\theta}{dt^2} + \omega^2 \theta &= 0 \label{eq:simplependulum2} \\
  \theta(t) &= A \cos(\omega t) + B \sin(\omega t) \label{eq:simplependulumsolution}
\end{align}

Eq.\eqref{eq:simplependulum1} and \eqref{eq:simplependulum2} are second order linear ordinary differential equation, whose solution Eq.\eqref{eq:simplependulumsolution} has 2 constants of integration $A$ and $B$. Here $\theta$ and $t$ are tge dependent and independent variable since Eq.\eqref{eq:simplependulum1} and \eqref{eq:simplependulum2} have only one independent variable.

Values of $A$ and $B$ will depend on initial condition.

Since ODE is second order, 2 initial condition are needed at initial time, say $t=0$. Which are:
\begin{align}
    \left.
        \begin{aligned}
	  \theta(t=0) = 1 \\
	  \frac{\theta(t=0)}{dt} = 0
        \end{aligned}
     \right\} \quad \quad & \label{eq:att=0}
\end{align}

Eq.\eqref{eq:simplependulum2} and initial conditions Eq.\eqref{eq:att=0} are together called \textbf{Mathematical IVP}. For any physical system the following two requirements are needed:
\begin{enumerate}[noitemsep]
	\item The equation (ODE or PDE) that governs the evolution of the above system.
	\item The initial state of the system.
\end{enumerate}

7 independent variables \textbf{(u,v,w,T,$\boldsymbol{\rho}$,p,q)}.

Surface area of Earth = $4\pi R^2$ = $4\pi (6.37 \times 10^{12})$ $\approx 5.1 \times 10^{14}$ m$^2$
%-------------------------------------------------------------------------------------
\end{document}
